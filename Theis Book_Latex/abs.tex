\addcontentsline{toc}{chapter}{Abstract}
\chapter*{\textbf{Abstract}}
In this project environment monitoring systems is implemented by using sensors and then sensors data are sent to a My SQL server using Wi-Fi. For enchanting data, DHT-11 (Temperature and Humidity sensor) and MQ-6 (LPG gas sensor) are being used. The basic objective of this research is to monitor and to develop a real-time monitoring of humidity and temperature, as well as the availability of gas using the very available DHT-11 sensor, MQ-6 sensor, and ESP-8266 NodeMCU module and then observe the data from a database. We can control the actuator from the server depending on the sensor value. Although great leap has been made in the control area, the precision motion control is challenging the control engineering to a greater extent. The control engineer needs to design a suitable controller which will effectively achieve the desired system characteristics, such as high precision, high speed requirements in precision motion control. here are two control schemes which have been proposed. In this research, both feedback and feedforward methods of control have been applied. This paper also makes a compact distinction between conventional and the local IP based observing system of an environment. The sensor's data are saved in a database by which we can monitor the sensor’s data without any access to the internet. In this research, the Arduino based ESP-8266 based NodeMCU was used. The various data attainment system of Arduino or Raspberry-pi is mother controller but using NodeMCU gives the benefit of using an Arduino along with a 2.4 GHz Wi-Fi module. As this was a demo project and needed far more inquiry in the real practice so, breadboards and jumper wires were used to test the project. 