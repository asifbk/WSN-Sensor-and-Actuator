\chapter{\textbf{Introduction}}
\section{Overview of Wireless Sensor Network}
Wireless sensor network (WSN) refers to a group of spatially dispersed and dedicated sensors for monitoring and recording the physical conditions of the environment and organizing the collected data at a central location. The propagation technique between the hops of the network can be routing or flooding. In computer science and telecommunications, wireless sensor networks are an active research area with numerous workshops and conferences arranged each year, for example IPSN, SenSys, and EWSN. Each such sensor network node has typically several parts: a radio transceiver with an internal antenna or connection to an external antenna, a microcontroller, an electronic circuit for interfacing with the sensors and an energy source, usually a battery or an embedded form of energy harvesting.\\\\
Enabled by recent advances in microelectronic mechanical systems (MEMS) and wireless communication technologies, tiny, cheap, and smart sensors deployed in a physical area and networked through wireless links and the Internet provide unprecedented opportunities for a variety of civilian and military applications, for example, environmental monitoring, battle field surveillance, and industry process control. Distinguished from traditional wireless communication networks, for example, cellular systems and mobile ad hoc networks (MANET), WSNs have unique characteristics, for example, denser level of node deployment, higher unreliability of sensor nodes, and severe energy, computation, and storage constraints, which present many new challenges in the development and application of WSNs. A large amount of research activities has been carried out to explore and solve various design and application issues, and significant advances have been made in the development and deployment of WSNs. It is envisioned that in the near future WSNs will be widely used in various civilian and military fields, and revolutionize the way we live, work, and interact with the physical world.

\section{Contribution and Motivation}
Safety is a less concerned issue in our country. Workers safety gets less priority and several fire accidents occur in the overcrowded rural areas in Bangladesh almost every year. No precautions are taken by the industries or the owners of the buildings as they don't want to pay extra for the safety issues. So, we tried to build a system which can monitor any system, warn people and can take actions if any accident especially the fire accident which would be not only effective but also cheap.\\\\
We have chosen Wireless Sensor Network Technology because of its flexibility and easy implementation capability. Moreover, we have attached a solar panel as a secondary power source to the system for the implementation of renewable energy. We believe our system is capable of after-effects of any kind fire accident and can monitor the humidity, temperature of the home, patients rooms or food or chemical industries where these factors should be strictly maintained.

\section{Characteristics of Wireless Sensor Network}
\begin{enumerate}[label=\roman*]
  \item  Due to the large number of sensor nodes, it is usually not possible to build a global addressing scheme for a sensor network because it would introduce a high overhead for the identification maintenance.
  \item  Once deployed, sensor nodes have to autonomously configure themselves into a communication network.
  \item	Sensor nodes are usually deployed in harsh or hostile environments and operate without attendance.
  \item	The number of sensor nodes in a sensor network can be several orders of magnitude higher than that in a MANET.
  \item	Sensor nodes are usually randomly deployed without careful planning and engineering.
  \item	Sensor nodes are highly limited in energy, computation, and storage capacities.
  \item	In most sensor network applications, the data sensed by sensor nodes flow from multiple source sensor nodes to a particular sink, exhibiting a many - to - one traffic pattern
  \item	In most sensor network applications, sensor nodes are densely deployed in a region of interest and collaborate to accomplish a common sensing task.
  \item	The data sensed by multiple sensor nodes typically have a certain level of correlation or redundancy.   
\end{enumerate}
%
%\begin{itemize}
%  \item Due to the large number of sensor nodes, it is usually not possible to build a global addressing scheme for a sensor network because it would introduce a high overhead for the identification maintenance.
%\end{itemize}

\section{Network Application}
WSNs were originally motivated by military applications, which range from large - scale acoustic surveillance systems for ocean surveillance to small networks of unattended ground sensors for ground target detection. Sensors can be used to detect or monitor a variety of physical parameters or conditions, for example:

\begin{enumerate}[label=\roman*]
  \item Light
\item Sound
\item Humility
\item Pressure
\item Temperature
\item Soil composition
\item Air or water quality
\item Attributes of an object such as size, weight, position, speed, and direction.
\end{enumerate}

They can not only reduce the cost and delay in deployment, but also be applied to any environment, especially those in which conventional wired sensor networks are impossible to be deployed, for example, inhospitable terrains, battlefields, outer space, or deep oceans. However, the availability of low - cost sensors and wireless communication has promised the development of a wide range of applications in both civilian and military fields.\\

(a) \textbf{Environment Monitoring}

In environmental monitoring, sensors are used to monitor a variety of environmental parameters or conditions. Environmental monitoring is one of the earliest applications of sensor networks. Sensors can be deployed on the ground or under water to monitor air or water quality. For example, water quality monitoring can be used in the hydrochemistry field. Sensors can be used to monitor biological or chemical hazards in locations, for example, a chemical plant or a battlefield. Sensors can be densely deployed in an intended region to detect natural or non - natural disasters. For example, sensors can be scattered in forests or revivers to detect forest fi res or floods Air or Water Quality Monitoring. from the University of California at Berkeley and the college of the Atlantic in Bar Harbor, conducted an experiment to monitor the habitat of the nesting petrels on Great Duck Land in Maine by deploying 190 wireless sensors, including humidity, pressure, temperature, and radiation.\\


 (b)	\textbf{Military Application}


Due to ease of deployment, self - configurability, untended operation, and fault tolerance, sensor networks will play more important roles in future military C3I systems and make future wars more intelligent with less human involvement. Sensors can be mounted on unmanned robotic vehicles, tanks, fighter planes, submarines, missiles, or torpedoes to guide them around obstacles to their targets and lead them to coordinate with one another to accomplish more effective attacks or defenses. Sensor nodes can be deployed around sensitive objects, for example, atomic plants, strategic bridges, oil and gas pipelines, communication centers, and military headquarters, for protection purpose. Sensors can be deployed for remote sensing of nuclear, biological, and chemical weapons, detection of potential terrorist attacks, and reconnaissance. Sensors can be deployed in a battlefield to monitor the presence of forces and vehicles, and track their movements, enabling close surveillance of opposing forces.

 (c)\textbf{Health Care Application}


WSNs can be used to monitor and track elders and patients for health care purposes, which can significantly relieve the severe shortage of health care personnel and reduce the health care expenditures in the current health care systems. Wearable sensors can be integrated into a wireless body area network (WBAN) to monitor vital signs, environmental parameters, and geographical locations, and thus allow long - term, noninvasive, and ambulatory monitoring of patients or elderly people with instantaneous alerts to health care personal in case of emergency, immediate reports to users about their current health statuses, and real - time updates of users ’ medical records.


 (d)\textbf{Industrial Process Control}


Tiny sensors can be embedded into the regions of a machine that are inaccessible by humans to monitor the condition of the machine and alert for any failure. For example, wireless sensors can be instrumented to production and assembly lines to monitor and control production processes. Chemical plants or oil refiners can use sensors to monitor the condition of their miles of pipelines. In industry, WSNs can be used to monitor manufacturing processes or the condition of manufacturing equipment.


 (e)\textbf{Security and Surveillance}


For example, acoustic, video, and other kinds of sensors can be deployed in buildings, airports, subways, and other critical infrastructure, for example, nuclear power plants or communication centers to identify and track intruders, and provide timely alarms and protection from potential attacks.

 (f)\textbf{Home Intelligence}


WSNs can be used to provide more convenient and intelligent living environments for human beings. Wireless sensors can be embedded into a home and connected to form an autonomous home network. Wireless sensors can be used to remotely read utility meters in a home, for example, water, gas, or electricity, and then send the readings to a remote center through wireless communication. In addition to the above applications, self - configurable WSNs can be used in many other areas, for example, disaster relief, traffic control, warehouse management, and civil engineering.

\section{Network Design Objectives}
\begin{justify}
(i)\textbf{Small Node Size:} Reducing node size is one of the primary design objectives of sensor networks. Sensor nodes are usually deployed in a harsh or hostile environment in large numbers. Reducing node size can facilitate node deployment, and also reduce the cost and power consumption of sensor nodes.\\
(ii)\textbf{Low Node Cost:} Reducing node cost is another primary design objective of sensor network. Since sensor nodes are usually deployed in a harsh or hostile environment in large numbers and cannot be reused, it is important to reduce the cost of sensor nodes so that the cost of the whole network is reduced.\\
(iii)\textbf{Low Power Consumption:} Reducing power consumption is the most important objective in the design of a sensor network. Since sensor nodes are powered by battery and it is often very difficult or even impossible to change or recharge their batteries, it is crucial to reduce the power consumption of sensor nodes so that the lifetime of the sensor nodes, as well as the whole network is prolonged.\\
(iv)\textbf{Self – Configurability:} In sensor networks, sensor nodes are usually deployed in a region of interest without careful planning and engineering. Once deployed, sensor nodes should be able to autonomously organize themselves into a communication network and reconfigure their connectivity in the event of topology changes and node failures.\\
(v)\textbf{Scalability:} In sensor networks, the number of sensor nodes may be on the order of tens, hundreds, or thousands. Thus, network protocols designed for sensor networks should be scalable to different network sizes.\\
(vi)\textbf{Adaptability:} In sensor networks, a node may fail, join, or move, which would result in changes in node density and network topology. Thus, network protocols designed for sensor networks should be adaptive to such density and topology changes.\\
(vii)\textbf{Reliability:} For many sensor network applications, it is required that data be reliably delivered over noisy, error - prone, and time - varying wireless channels. To meet this requirement, network protocols designed for sensor networks must provide error control and correction mechanisms to ensure reliable data delivery.\\
(viii)\textbf{Fault Tolerance:} Sensor nodes are prone to failures due to harsh deployment environments and unattended operations. Thus, sensor nodes should be fault tolerant and have the abilities of self - testing, self - calibrating, self-repairing, and self - recovering.\\
(ix)\textbf{Security:} In many military applications, sensor nodes are deployed in a hostile environment and thus are vulnerable to adversaries. In such situations, a sensor network should introduce effective security mechanisms to prevent the data information in the network or a sensor node from unauthorized access or malicious attacks.\\
(x)\textbf{Channel Utilization:} Sensor networks have limited bandwidth resources. Thus, communication protocols designed for sensor networks should efficiently make use of the bandwidth to improve channel utilization.\\
(xi)\textbf{QoS Support:} In sensor networks, different applications may have different quality - of - service (QoS) requirements in terms of delivery latency and packet loss. For example, some applications, for example, fi re monitoring, are delay sensitive and thus require timely data delivery. Some applications, for example, data collection for scientific exploration, are delay tolerant but cannot stand packet loss. Thus, network protocol design should consider the QoS requirements of specific applications.\\
\end{justify}

\section{Network Design Challenges}
\begin{justify}
(i)	\textbf{Limited Hardware Resources:} Sensor nodes have limited processing and storage capacities, and thus can only perform limited computational functionalities. These hardware constraints present many challenges in software development and network protocol design for sensor networks, which must consider not only the energy constraint in sensor nodes, but also the processing and storage capacities of sensor nodes.\\
(ii)\textbf{Massive and Random Deployment:} Most sensor networks consist of a large number of sensor nodes, from hundreds to thousands or even more. Node deployment is usually application dependent, which can be either manual or random. In most applications, sensor nodes can be scattered randomly in an intended area or dropped massively over an inaccessible or hostile region. The sensor nodes must autonomously organize themselves into a communication network before they start to perform a sensing task.\\
(iii)\textbf{Dynamic and Unreliable Environment:} A sensor network usually operates in a dynamic and unreliable environment. On one hand, the topology of a sensor network may change frequently due to node failures, damages, additions, or energy depletion. On the other hand, sensor nodes are linked by a wireless medium, which is noisy, error prone, and time varying. The connectivity of the network may be frequently disrupted because of channel fading or signal attenuation.\\
(iv)\textbf{Diverse Application:} Sensor networks have a wide range of diverse applications. The requirements for different applications may vary significantly. No network protocol can meet the requirements of all applications. The design of sensor networks is application specific.\\
(v)\textbf{Limited Energy Capacity:} Sensor nodes are battery powered and thus have very limited energy capacity. This constraint presents many new challenges in the development of hardware and software, and the design of network architectures and protocols for sensor networks. To prolong the operational lifetime of a sensor network, energy efficiency should be considered in every aspect of sensor network design, not only hardware and software, but also network architectures and protocols.\\
\end{justify}

\section{Wireless Communication Technology}
At higher layers, efficient communication protocols have been developed to address various networking issues, for example, medium access control, routing, QoS, and network security. Wireless communication has been extensively studied for conventional wireless networks in the last couple of decades and significant advances have been obtained in various aspects of wireless communication. Today most conventional wireless networks use radio frequency (RF) for communication, including microwave and millimeter wave. However, RF has some limitations, for example, large radiators and low transmission efficiencies, which make RF not the best communication medium for tiny energy - constrained sensor nodes. On the other hand, most communication protocols for conventional wireless networks, for example, cellular systems, wireless local area networks (WLANs), wireless personal area networks (WPANs), and MANETs, do not consider the unique characteristics of sensor networks, in particular, the energy constraint in sensor nodes. The high directivity of optical communication enables the use of spatial division multiple access (SDMA), which requires no communication overhead and has the potential to be more energy efficient than the medium access schemes used in RF, such as time, frequency, and code division multiple access (TDMA, FDMA, and CDMA).\\\\
It has been shown that DVS based power management has significantly higher energy efficiency compared to shut down - based power management. Meanwhile, power consumption can further be reduced through efficiently operating various system resources using some dynamic power management (DPM) technique. On the other hand, energy efficiency can significantly be enhanced if energy awareness is incorporated in the design of system software, including the operating system, and application and network protocols. To achieve low - power consumption at the node level, it is necessary to incorporate power awareness and energy optimization in hardware design for sensor networks. For example, a commonly used DPM technique is to shutdown idle components or put them in a low - power state when there is little or no load to process, which can significantly reduce power consumption. Low - power circuit and system design has enabled the development of ultralow power hardware components, for example, microprocessors and microcontrollers.

\subsection{Hardware Platform}
This class of platforms include the Berkeley mote family, the UCLA Medusa family, and MIT $\mu$ AMP, which typically use commercial off - the - shelf chips and are characterized by small form factors, low - power processing and communication, and simple sensor interfaces. Compared with dedicated and SoC sensor nodes, these PC - like platforms have higher processing capability and thus can incorporate a richer set of networking protocols, popular programming languages, middleware, application programming interfaces (APIs), and other off - the - shelf software. This class of platforms include various low - power embedded PCs (e.g., PC104) and personal digital assistants (PDAs), which typically run off - the - shelf operating systems, for example, Win CE, Linux, or real - time operating systems, and use standard wireless communication protocols, for example, IEEE 802.11 or Bluetooth. Sensor node hardware platforms can be classified into three categories: augmented general - purpose personal computers (PCs), dedicated sensor nodes, and system - on - chip (SoC) sensor nodes.

\subsection{Software Platform}
A software platform can be an operating system that provides a set of services for applications, including fi le management, memory allocation, task scheduling, peripheral device drivers, and networking, or it can be a language platform that provides a library of components to programmers. Tiny OS is one of the earliest operating systems supporting sensor network applications on resource constrained hardware platforms, for example, the Berkeley motes. It defines virtual machine instructions to abstract those common operations, for example, polling sensors and accessing internal states. Tiny GALS is a language for Tiny OS, which provides a way of building event - triggered concurrent execution from thread - unsafe components. Therefore, software written in Mot é instructions does not have to be rewritten to accommodate a new hardware platform with support for the virtual machine. It provides a set of language constructs and restrictions to implement Tiny OS components and applications.\\\\
\textbf{The IEEE 802.15.4 Standard:} The IEEE 802.15.4 is a standard developed by IEEE 802.15 Task Group 4, which specifies the physical and MAC layers for low - rate WPANs. As defined in its Project Authorization Request, the goal of Task Group 4 is to “provide a standard for ultralow complexity, ultralow cost, ultralow power consumption, and low - data rate wireless connectivity among inexpensive devices”. The first release of the IEEE 802.15.4 standard was delivered in 2003 and is freely distributed. This release was revised in 2006, but the new release is not yet freely distributed. Its protocol stack is simple and flexible, and does not require any infrastructure. The standard has the following features:

\begin{enumerate}[label=\roman*]
  \item	Data rates of 250 kbps, 40 kbps, and 20 kbps.
  \item	Two addressing modes: 16 - bit short and 64 - bit IEEE addressing. 
  \item Support for critical latency devices, for example, joysticks.
  \item	The CSMA - CA channel access.
  \item	Automatic network establishment by the coordinator.
  \item	Fully handshaking protocol for transfer reliability.
  \item	Power management to ensure low - power consumption.
  \item	Some 16 channels in the 2.4 - GHz ISM band, 10 channels in the 915 - MHz band, and 1 channel in the 868 - MHz band.
\end{enumerate}

\textbf{The ZigBee Standard:} The IEEE 802.15.4 standard only defines the physical and MAC layers without specifying the higher protocol layers, including the network and application layers. The ZigBee standard is developed on top of the IEEE 802.15.4 standard and defines the network and application layers. The network layer provides networking functionalities for different network topologies, and the application layer provides a framework for distributed application development and communication. The two protocol stacks can be combined together to support short - range low data rate wireless communication with battery - powered wireless devices. The potential applications of these standards include sensors, interactive toys, smart badges, remote controls, and home automation. The ZigBee protocol stack was proposed at the end of 2004 by the ZigBee Alliance, an association of companies working together to enable reliable, cost - effective, low - power, wirelessly networked, monitoring, and control products based on an open global standard. The first release of ZigBee was revised at the end of 2006, which introduces extensions on the standardization of application profiles and some minor improvements to the network and application layers.\\\\
\textbf{The IEEE 1451 Standard:} The IEEE 1451 standards are a family of Smart Transducer Interface Standards that defines a set of open, common, network - independent communication interfaces for connecting transducers (i.e., sensors or actuators) to microprocessors, instrumentation systems, and control/ field networks. Transducers have a wide variety of applications in industry, for example, manufacturing, industrial control, automotive, aerospace, building, biomedicine
\textbf{IEEE 1451 standards:}
\begin{enumerate}[label=\roman*]
  \item	IEEE P1451.0 defines a set of common commands, common operations, and TEDS for the family of IEEE 1451 smart transducer standards. Through this command set, one can access any sensors or actuators in the 1451 - based wired and wireless networks.
      \item	IEEE 1451.1 defines a common object model describing the behavior of smart transducers, a measurement model that streamlines measurement processes, and the communication models used for the standard, which includes the client - server and publish - subscribe models.
          \item	IEEE 1451.2 defines a transducer - to - NCAP interface and TEDS for a point - to - point configuration.
          \item	IEEE 1451.3 defines a transducer - to - NCAP interface and TEDS for multidrop transducers using a distributed communication architecture. It allowed many transducers to be arrayed as nodes, on a multidrop transducer network, sharing a common pair of wires.
\item	IEEE 1451.4 defines a mixed - mode interface for analog transducers with analog and digital operating modes.
\item	IEEE P1451.5 defines a transducer - to - NCAP interface and TEDS for wireless transducers. Protocol standards for wireless networks, for example, 802.11 (WIFI), 802.15.1 (Bluetooth), and 802.15.4 (ZigBee), are being considered as some of the physical interfaces for IEEE P1451.5.
\end{enumerate}
